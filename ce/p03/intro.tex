\section{Introdução}

As aplicações de semicondutores no campo de controle de potência foram importantíssimas para o desenvolvimento vertiginoso de dispositivos mais confiáveis. Esses dispositivos foram largamente empregados na execução de tarefas cotidianas como a regulagem do aquecimento de um chuveiro elétrico e da iluminação das residências e em atividades mais críticas como em sistemas de transmissão DC (\textit{Direct Current}) ou AC (\textit{Alternating Current}) de alta tensão e motores acionadores (DC ou AC) \cite{ashfaq2000}.

Esse panorama atual e, por conseguinte, a denominação desse campo de estudo com o termo \textit{Eletrônica de Potência} tem início na década de 60 logo após o surgimento de um dispositivo denominado SCR (\textit{Silicon Controlled Rectifier}). Esse dispositivo apresentava vantagens de pequeno porte, baixo custo, e alta eficiência na transferência de potência. Dessa forma, pesquisadores e fabricantes, debruçaram-se suas atenções para a esse novo campo permitindo a busca de novos componentes ainda mais eficientes e progresso da Eletrônica de Potência exponencialmente.